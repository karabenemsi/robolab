% !TEX root = Document.tex
\section*{Task 1}

The constructed generator matrix $G'_{4,7}$ and the corresponding parity-check matrix $H'_{3,7}$ without extension look like this:

\begin{align*}
  G'_{4,7}             & := \begin{pNiceMatrix}[first-row]
                              p_1 & p_2 & d_1 & p_3 & d_2 & d_3 & d_4 \\
                              1   & 1   & 1   & 0   & 0   & 0   & 0   \\
                              1   & 0   & 0   & 1   & 1   & 0   & 0   \\
                              0   & 1   & 0   & 1   & 0   & 1   & 0   \\
                              1   & 1   & 0   & 1   & 0   & 0   & 1   \\
                            \end{pNiceMatrix} \\
  \Rightarrow A^T      & = \begin{pNiceMatrix}[]
                             1 & 1 & 0 \\
                             1 & 0 & 1 \\
                             0 & 1 & 1 \\
                             1 & 1 & 1 \\
                           \end{pNiceMatrix}
  \rightarrow A    = \begin{pNiceMatrix}[]
                       1 & 1 & 0 & 1 \\
                       1 & 0 & 1 & 1 \\
                       0 & 1 & 1 & 1 \\
                     \end{pNiceMatrix}                           \\
  \Rightarrow H'_{3,7} & := \begin{pNiceMatrix}[first-row]
                              p_1 & p_2 & d_1 & p_3 & d_2 & d_3 & d_4 \\
                              1   & 0   & 1   & 0   & 1   & 0   & 1   \\
                              0   & 1   & 1   & 0   & 0   & 1   & 1   \\
                              0   & 0   & 0   & 1   & 1   & 1   & 1   \\
                            \end{pNiceMatrix}
\end{align*}


Adding the additional parity bit $p_4$ for the extended Hamming Code for the generator matrix $G_{4,8}$ looks like this:

\begin{equation*}
  G'_{4,8} := \begin{pNiceMatrix}[first-row]
    p_1 & p_2 & d_1 & p_3 & d_2 & d_3 & d_4 & p_4 \\
    1   & 1   & 1   & 0   & 0   & 0   & 0   & 1   \\
    1   & 0   & 0   & 1   & 1   & 0   & 0   & 1   \\
    0   & 1   & 0   & 1   & 0   & 1   & 0   & 1   \\
    1   & 1   & 0   & 1   & 0   & 0   & 1   & 0   \\
  \end{pNiceMatrix} \\
\end{equation*}

And the corresponding parity-check matrix $H_{4,8}$ looks like this:

\begin{equation*}
  \Rightarrow H'_{4,8} := \begin{pNiceMatrix}
    [first-row]
    p_1 & p_2 & d_1 & p_3 & d_2 & d_3 & d_4 & p_4 \\
    1   & 0   & 1   & 0   & 1   & 0   & 1   & 0   \\
    0   & 1   & 1   & 0   & 0   & 1   & 1   & 0   \\
    0   & 0   & 0   & 1   & 1   & 1   & 1   & 0   \\
    1   & 1   & 1   & 1   & 1   & 1   & 1   & 1   \\
  \end{pNiceMatrix}
\end{equation*}

\section*{Task 2}
\begin{align*}
  G'_{4,8} := \begin{pNiceMatrix}[first-row]
                p_1 & p_2 & d_1 & p_3 & d_2 & d_3 & d_4 & p_4 \\
                1   & 1   & 1   & 0   & 0   & 0   & 0   & 1   \\
                1   & 0   & 0   & 1   & 1   & 0   & 0   & 1   \\
                0   & 1   & 0   & 1   & 0   & 1   & 0   & 1   \\
                1   & 1   & 0   & 1   & 0   & 0   & 1   & 0   \\
              \end{pNiceMatrix} \\
\end{align*}
To convert this non-systematic generator matrix into a systematic one, we have to apply the following steps:
\begin{align*}
  \text{Step 1:}      &                                                  & \\
                      & \begin{pNiceMatrix}[first-col]
                                    & 1 & 1 & 1 & 0 & 0 & 0 & 0 & 1 \\
                          r_2 + r_1 & 0 & 1 & 1 & 1 & 1 & 0 & 0 & 0 \\
                                    & 0 & 1 & 0 & 1 & 0 & 1 & 0 & 1 \\
                          r_4 + r_1 & 0 & 0 & 1 & 1 & 0 & 0 & 1 & 1 \\
                        \end{pNiceMatrix}     &          \\
  \text{Step 2:}      &                                                  & \\
                      & \begin{pNiceMatrix}[first-col]
                          r_1 +r_2  & 1 & 0 & 0 & 1 & 1 & 0 & 0 & 1 \\
                                    & 0 & 1 & 1 & 1 & 1 & 0 & 0 & 0 \\
                          r_3 + r_2 & 0 & 0 & 1 & 0 & 1 & 1 & 0 & 1 \\
                                    & 0 & 0 & 1 & 1 & 0 & 0 & 1 & 1 \\
                        \end{pNiceMatrix}     &          \\
  \text{Step 3:}      &                                                  & \\
                      & \begin{pNiceMatrix}[first-col]
                                    & 1 & 0 & 0 & 1 & 1 & 0 & 0 & 1 \\
                          r_2 + r_3 & 0 & 1 & 0 & 1 & 0 & 1 & 0 & 1 \\
                                    & 0 & 0 & 1 & 0 & 1 & 1 & 0 & 1 \\
                          r_4 + r_3 & 0 & 0 & 0 & 1 & 1 & 1 & 1 & 0 \\
                        \end{pNiceMatrix}     &          \\
  \text{Step 4:}      &                                                  & \\
                      & \begin{pNiceMatrix}[first-col]
                          r_1 + r_4 & 1 & 0 & 0 & 0 & 0 & 1 & 1 & 1 \\
                          r_2 + r_4 & 0 & 1 & 0 & 0 & 1 & 0 & 1 & 1 \\
                                    & 0 & 0 & 1 & 0 & 1 & 1 & 0 & 1 \\
                                    & 0 & 0 & 0 & 1 & 1 & 1 & 1 & 0 \\
                        \end{pNiceMatrix}          \\
  \Rightarrow G_{4,8} & \begin{pNiceMatrix}[first-row]
                          d_1 & d_2 & d_3 & d_4 & p_1 & p_2 & p_3 & p_4 \\
                          1   & 0   & 0   & 0   & 0   & 1   & 1   & 1   \\
                          0   & 1   & 0   & 0   & 1   & 0   & 1   & 1   \\
                          0   & 0   & 1   & 0   & 1   & 1   & 0   & 1   \\
                          0   & 0   & 0   & 1   & 1   & 1   & 1   & 0   \\
                        \end{pNiceMatrix}      \\
\end{align*}

As G is defined as $G:=(I_k|-A^T)$, we can read the parity-check matrix $H$ directly from the matrix $A$: $H:=(-A|I_{n-k})$:
\begin{equation*}
  H_{4,8} := \begin{pNiceMatrix}[first-row]
    d_1 & d_2 & d_3 & d_4 & p_1 & p_2 & p_3 & p_4 \\
    0   & 1   & 1   & 1   & 1   & 0   & 0   & 0   \\
    1   & 0   & 1   & 1   & 0   & 1   & 0   & 0   \\
    1   & 1   & 0   & 1   & 0   & 0   & 1   & 0   \\
    1   & 1   & 1   & 0   & 0   & 0   & 0   & 1   \\
  \end{pNiceMatrix}
\end{equation*}

\section*{Task 3}
\begin{align*}
  \vec{x_1} & = (\vec{a_1} \cdot G') \mod 2 = (0100) \cdot G' \mod 2 = (1 0 0 1 1 0 0 1) \mod 2 = (1 0 0 1 1 0 0 1) \\
  \vec{x_2} & = (\vec{a_2} \cdot G') \mod 2 = (22110011) \mod 2 = (00110011)                                        \\
  \vec{x_3} & = (\vec{a_3} \cdot G') \mod 2 = (10000111)                                                            \\
  \vec{x_4} & = (\vec{a_4} \cdot G') \mod 2 = (10101010)                                                            \\
\end{align*}

\section*{Task 4}