% !TEX root = Document.tex
\section*{Task 1}

The constructed generator matrix $G'_{4,7}$ and the corresponding parity-check matrix $H'_{3,7}$ without extension look like this:

\begin{align*}
  G'_{4,7}             & := \begin{pNiceMatrix}[first-row]
                              p_1 & p_2 & d_1 & p_3 & d_2 & d_3 & d_4 \\
                              1   & 1   & 1   & 0   & 0   & 0   & 0   \\
                              1   & 0   & 0   & 1   & 1   & 0   & 0   \\
                              0   & 1   & 0   & 1   & 0   & 1   & 0   \\
                              1   & 1   & 0   & 1   & 0   & 0   & 1   \\
                            \end{pNiceMatrix} \\
  \Rightarrow A^T      & = \begin{pNiceMatrix}[]
                             1 & 1 & 0 \\
                             1 & 0 & 1 \\
                             0 & 1 & 1 \\
                             1 & 1 & 1 \\
                           \end{pNiceMatrix}
  \rightarrow A    = \begin{pNiceMatrix}[]
                       1 & 1 & 0 & 1 \\
                       1 & 0 & 1 & 1 \\
                       0 & 1 & 1 & 1 \\
                     \end{pNiceMatrix}                           \\
  \Rightarrow H'_{3,7} & := \begin{pNiceMatrix}[first-row]
                              p_1 & p_2 & d_1 & p_3 & d_2 & d_3 & d_4 \\
                              1   & 0   & 1   & 0   & 1   & 0   & 1   \\
                              0   & 1   & 1   & 0   & 0   & 1   & 1   \\
                              0   & 0   & 0   & 1   & 1   & 1   & 1   \\
                            \end{pNiceMatrix}
\end{align*}


Adding the additional parity bit $p_4$ for the extended Hamming Code for the generator matrix $G_{4,8}$ looks like this:

\begin{equation*}
  G'_{4,8} := \begin{pNiceMatrix}[first-row]
    p_1 & p_2 & d_1 & p_3 & d_2 & d_3 & d_4 & p_4 \\
    1   & 1   & 1   & 0   & 0   & 0   & 0   & 1   \\
    1   & 0   & 0   & 1   & 1   & 0   & 0   & 1   \\
    0   & 1   & 0   & 1   & 0   & 1   & 0   & 1   \\
    1   & 1   & 0   & 1   & 0   & 0   & 1   & 0   \\
  \end{pNiceMatrix} \\
\end{equation*}

And the corresponding parity-check matrix $H_{4,8}$ looks like this:

\begin{equation*}
  \Rightarrow H'_{4,8} := \begin{pNiceMatrix}[first-row]
    p_1 & p_2 & d_1 & p_3 & d_2 & d_3 & d_4 & p_4 \\
    1   & 0   & 1   & 0   & 1   & 0   & 1   & 0   \\
    0   & 1   & 1   & 0   & 0   & 1   & 1   & 0   \\
    0   & 0   & 0   & 1   & 1   & 1   & 1   & 0   \\
    1   & 1   & 1   & 1   & 1   & 1   & 1   & 1   \\
  \end{pNiceMatrix}
\end{equation*}

\section*{Task 2}
\begin{align*}
  G'_{4,8} := \begin{pNiceMatrix}[first-row]
                p_1 & p_2 & d_1 & p_3 & d_2 & d_3 & d_4 & p_4 \\
                1   & 1   & 1   & 0   & 0   & 0   & 0   & 1   \\
                1   & 0   & 0   & 1   & 1   & 0   & 0   & 1   \\
                0   & 1   & 0   & 1   & 0   & 1   & 0   & 1   \\
                1   & 1   & 0   & 1   & 0   & 0   & 1   & 0   \\
              \end{pNiceMatrix} \\
\end{align*}
To convert this non-systematic generator matrix into a systematic one, we have to apply the following steps:
\begin{align*}
  \text{Step 1:}      &                                                  & \\
                      & \begin{pNiceMatrix}[first-col]
                                    & 1 & 1 & 1 & 0 & 0 & 0 & 0 & 1 \\
                          r_2 + r_1 & 0 & 1 & 1 & 1 & 1 & 0 & 0 & 0 \\
                                    & 0 & 1 & 0 & 1 & 0 & 1 & 0 & 1 \\
                          r_4 + r_1 & 0 & 0 & 1 & 1 & 0 & 0 & 1 & 1 \\
                        \end{pNiceMatrix}     &          \\
  \text{Step 2:}      &                                                  & \\
                      & \begin{pNiceMatrix}[first-col]
                          r_1 +r_2  & 1 & 0 & 0 & 1 & 1 & 0 & 0 & 1 \\
                                    & 0 & 1 & 1 & 1 & 1 & 0 & 0 & 0 \\
                          r_3 + r_2 & 0 & 0 & 1 & 0 & 1 & 1 & 0 & 1 \\
                                    & 0 & 0 & 1 & 1 & 0 & 0 & 1 & 1 \\
                        \end{pNiceMatrix}     &          \\
  \text{Step 3:}      &                                                  & \\
                      & \begin{pNiceMatrix}[first-col]
                                    & 1 & 0 & 0 & 1 & 1 & 0 & 0 & 1 \\
                          r_2 + r_3 & 0 & 1 & 0 & 1 & 0 & 1 & 0 & 1 \\
                                    & 0 & 0 & 1 & 0 & 1 & 1 & 0 & 1 \\
                          r_4 + r_3 & 0 & 0 & 0 & 1 & 1 & 1 & 1 & 0 \\
                        \end{pNiceMatrix}     &          \\
  \text{Step 4:}      &                                                  & \\
                      & \begin{pNiceMatrix}[first-col]
                          r_1 + r_4 & 1 & 0 & 0 & 0 & 0 & 1 & 1 & 1 \\
                          r_2 + r_4 & 0 & 1 & 0 & 0 & 1 & 0 & 1 & 1 \\
                                    & 0 & 0 & 1 & 0 & 1 & 1 & 0 & 1 \\
                                    & 0 & 0 & 0 & 1 & 1 & 1 & 1 & 0 \\
                        \end{pNiceMatrix}          \\
  \Rightarrow G_{4,8} & \begin{pNiceMatrix}[first-row]
                          d_1 & d_2 & d_3 & d_4 & p_1 & p_2 & p_3 & p_4 \\
                          1   & 0   & 0   & 0   & 0   & 1   & 1   & 1   \\
                          0   & 1   & 0   & 0   & 1   & 0   & 1   & 1   \\
                          0   & 0   & 1   & 0   & 1   & 1   & 0   & 1   \\
                          0   & 0   & 0   & 1   & 1   & 1   & 1   & 0   \\
                        \end{pNiceMatrix}      \\
\end{align*}

As G is defined as $G:=(I_k|-A^T)$, we can read the parity-check matrix $H$ directly from the matrix $A$: $H:=(-A|I_{n-k})$ and add the parity bit row to the end of the matrix:
\begin{equation*}
  H_{4,8} := \begin{pNiceMatrix}[first-row]
    d_1 & d_2 & d_3 & d_4 & p_1 & p_2 & p_3 & p_4 \\
    0   & 1   & 1   & 1   & 1   & 0   & 0   & 0   \\
    1   & 0   & 1   & 1   & 0   & 1   & 0   & 0   \\
    1   & 1   & 0   & 1   & 0   & 0   & 1   & 0   \\
    1   & 1   & 1   & 1   & 1   & 1   & 1   & 1   \\
  \end{pNiceMatrix}
\end{equation*}

\section*{Task 3}
\begin{align*}
  \vec{x_1} & = (\vec{a_1} \cdot G') \mod 2 = (0100) \cdot G' \mod 2 = (1 0 0 1 1 0 0 1) \mod 2 = (1 0 0 1 1 0 0 1) \,  p_4 = 1\\
  \vec{x_2} & = (\vec{a_2} \cdot G') \mod 2= (1001) \cdot G' \mod 2 = = (22110011) \mod 2 = (00110011) \, p_4 = 1                \\
  \vec{x_3} & = (\vec{a_3} \cdot G') \mod 2 = (0011) \cdot G' \mod 2 = = (10000111) \, p_4 = 1                                                            \\
  \vec{x_4} & = (\vec{a_4} \cdot G') \mod 2 = (1101) \cdot G' \mod 2 = = (10101010)  \, p_4 = 0                                                            \\
\end{align*}

\section*{Task 4}
\begin{align*}
  \vec{x_1} & = (1 1 0 0 1 1 0 1)\\
  \vec{x_2} & = (1 0 0 1 1 0 0 1)           \\
  \vec{x_3} & = (1 1 0 1 1 0 1 1)                                                          \\
  \vec{x_4} & = (1 1 0 1 0 1 0 1)                                                         \\
\end{align*}

\subsection*{Parity bit $p_4$}
\begin{align*}
  \vec{x_1} & = (1 1 0 0 1 1 0 1) \rightarrow p_4 = 1 \rightarrow \text{error}\\
  \vec{x_2} & = (1 0 0 1 1 0 0 1)  \rightarrow p_4 = 1 \rightarrow \text{correct}         \\
  \vec{x_3} & = (1 1 0 1 1 0 1 1)  \rightarrow p_4 = 1 \rightarrow \text{correct}                                                        \\
  \vec{x_4} & = (1 1 0 1 0 1 0 1)   \rightarrow p_4 = 1 \rightarrow \text{error}                                                      \\
\end{align*}

\subsection*{Syndrome vector}
The syndrome vector $\vec{z}$ is calculated by multiplying the received vector $\vec{x}$ with the parity-check matrix $H$, where the last bit is the parity bit $p_4$ and the first three bits are the syndrome vector $\vec{z}$:
\begin{equation*}
  \vec{z} = (\vec{x} \cdot H) \mod 2
\end{equation*}
This results in the following syndrome vectors:
\begin{align*}
  \vec{z_1} & = (\vec{x_1} \cdot H) \mod 2 = ((11001101) \cdot H) \mod 2\\
   &= \begin{pmatrix}2 &2 &2 &5\end{pmatrix} \mod 2 = \begin{pmatrix}0 &0 &0 &1\end{pmatrix}\\
   & \Rightarrow \text{error in parity bit}\\
  \vec{z_2} & = (\vec{x_2} \cdot H) \mod 2 = ((10011001) \cdot H) \mod 2\\
   &= \begin{pmatrix}2 &2 &2 &4\end{pmatrix} \mod 2 = \begin{pmatrix}0 &0 &0 &0\end{pmatrix}\\
   & \Rightarrow \text{no error}\\
  \vec{z_3} & = (\vec{x_3} \cdot H) \mod 2 = ((11011011) \cdot H) \mod 2\\
   &= \begin{pmatrix}3 &2 &4 &6\end{pmatrix} \mod 2 = \begin{pmatrix}1 &0 &0 &0\end{pmatrix}\\
   & \Rightarrow \text{multiple errors}\\
  \vec{z_4} & = (\vec{x_4} \cdot H) \mod 2 = ((11010101) \cdot H) \mod 2\\
   &= \begin{pmatrix}2 &3 &3 &5\end{pmatrix} \mod 2 = \begin{pmatrix}0 &1 &1 &1\end{pmatrix}\\
   & \Rightarrow \text{error, try correction}\\
\end{align*}

\subsection*{Error correction}
For error correction, we have to find the syndrome $\vec{z_4}$ as a column in $H$ and flip the bit in the received vector $\vec{x_4}$ at the matching column:

The first column of $H$ is $\begin{pmatrix}0 &1 &1 &1\end{pmatrix}^T$, and matches the syndrome $\vec{z_4} = \begin{pmatrix}0 &1 &1 &1\end{pmatrix}$ so we have to flip the bit at the first position in $\vec{x_4}$:


\begin{align*}
  \vec{x_4} = (1 1 0 1 0 1 0 1) \rightarrow \text{flip bit at position 0} \rightarrow \vec{x_4'} = (0 1 0 1 0 1 0 1)\\
\end{align*}
To validate the correction, we calculate the syndrome vector again:
\begin{align*}
  \vec{z_4'} & = (\vec{x_4'} \cdot H) \mod 2 = ((01010101) \cdot H) \mod 2\\
   &= \begin{pmatrix}0 &0 &0 &0\end{pmatrix}
    \Rightarrow \text{no error}
\end{align*}


\section*{Appendix}
\subsection*{Python code for encoding and decoding}
\lstinputlisting[language=Python]{./code/hc.py}

