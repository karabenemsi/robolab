% !TEX root = ../Document.tex


\setlength{\topskip}{\ht\strutbox}
\geometry{a4paper,left=25mm,right=25mm,top=30mm,bottom=30mm}

\usepackage[
	automark, % Kapitelangaben in Kopfzeile automatisch erstellen
	headsepline, % Trennlinie unter Kopfzeile
	ilines % Trennlinie linksbündig ausrichten
]{scrlayer-scrpage}

\pagestyle{scrheadings}
% chapterpagestyle gibt es nicht in scrartcl
%\renewcommand{\chapterpagestyle}{scrheadings}
\clearpairofpagestyles


% Kopfzeile
\renewcommand{\headfont}{\normalfont} % Schriftform der Kopfzeile
\ihead{\textsc{\docutitle}\\[0.5ex] \textit{\headmark}}
\chead{}
\ohead{\includegraphics[scale=0.5]{\logo}}
\setlength{\headheight}{15mm} % Höhe der Kopfzeile
%\setheadwidth[0pt]{textwithmarginpar} % Kopfzeile über den Text hinaus verbreitern (falls Logo den Text überdeckt)

% Fußzeile
\ifoot{\today}
\cfoot{}
\ofoot{\pagemark}
\setlength{\footskip}{10mm}

% Überschriften nach DIN 5008 in einer Fluchtlinie
% ------------------------------------------------------------------------------

% Abstand zwischen Nummerierung und Überschrift definieren
% > Schön wäre hier die dynamische Berechnung des Abstandes in Abhängigkeit
% > der Verschachtelungstiefe des Inhaltsverzeichnisses
\newcommand{\headingSpace}{1.5cm}

% Abschnittsüberschriften im selben Stil wie beim Inhaltsverzeichnis einrücken
\renewcommand*{\othersectionlevelsformat}[3]{
  \makebox[\headingSpace][l]{#3\autodot}
}


% Allgemeines
% ------------------------------------------------------------------------------

\onehalfspacing
\frenchspacing

% Schusterjungen und Hurenkinder vermeiden
\clubpenalty = 10000
\widowpenalty = 10000
\displaywidowpenalty = 10000

% Quellcode-Ausgabe formatieren
%\lstset{numbers=left, numberstyle=\tiny, numbersep=5pt, breaklines=true}
\lstset{emph={square}, emphstyle=\color{red}, emph={[2]root,base}, emphstyle={[2]\color{blue}}}

\counterwithout{footnote}{section} % Fußnoten fortlaufend durchnummerieren

% Aufzählungen anpassen
\renewcommand{\labelenumi}{\arabic{enumi}.}
\renewcommand{\labelenumii}{\arabic{enumi}.\arabic{enumii}.}
\renewcommand{\labelenumiii}{\arabic{enumi}.\arabic{enumii}.\arabic{enumiii}}

% Tabellenfärbung:
\definecolor{heading}{rgb}{0.64,0.78,0.86}
\definecolor{odd}{rgb}{0.9,0.9,0.9}

% Schriftart
\usepackage[sfdefault, scale=1, light]{roboto}  % Roboto als Schrifart
\usepackage[T1]{fontenc}
%\usepackage{newtxsf}
